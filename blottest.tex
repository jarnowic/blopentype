\input blot
\pdfpageheight 12 true in
\vsize10in

\hsize6.5in
\footline{}

\setfont\mainfont :
  name = "Palatino Linotype"
  size = 12pt
  big  = +2pt

\baselineskip=1.5\baselineskip

\patterns{1τ}

\bf Two statements by Lincoln

\rg

Address delivered at the dedication of the cemetery at Gettysburg.

November 19, 1863.

Four score and seven years ago our fathers brought forth on this continent, a new nation, conceived in Liberty, and dedicated to the proposition that all men are created equal.

Now we are engaged in a great civil war, testing whether that nation, or any nation so conceived and so dedicated, can long endure. We are met on a great battle-field of that war. We have come to dedicate a portion of that field, as a final resting place for those who here gave their lives that that nation might live. It is altogether fitting and proper that we should do this.

But, in a larger sense, we can not dedicate—we can not consecrate—we can not hallow—this ground. The brave men, living and dead, who struggled here, have consecrated it, far above our poor power to add or detract. The world will little note, nor long remember what we say here, but it can never forget what they did here. It is for us the living, rather, to be dedicated here to the unfinished work which they who fought here have thus far so nobly advanced. It is rather for us to be here dedicated to the great task remaining before us—that from these honored dead we take increased devotion to that cause for which they gave the last full measure of devotion—that we here highly resolve that these dead shall not have died in vain—that this nation, under God, shall have a new birth of freedom—and that government of the people, by the people, for the people, shall not perish from the earth.

—Abraham Lincoln

\vskip\baselineskip

Executive Mansion,

Washington, Nov. 21, 1864.

Dear Madam,

I have been shown in the files of the War Department a statement of the Adjutant General of Massachusetts that you are the mother of five sons who have died gloriously on the field of battle.

I feel how weak and fruitless must be any words of mine which should attempt to beguile you from the grief of a loss so overwhelming. But I cannot refrain from tendering to you the consolation that may be found in the thanks of the Republic they died to save.

I pray that our Heavenly Father may assuage the anguish of your bereavement, and leave you only the cherished memory of the loved and lost, and the solemn pride that must be yours to have laid so costly a sacrifice upon the altar of Freedom.

Yours, very sincerely and respectfully,

A. Lincoln.

Mrs. [Lydia Parker] Bixby.

[Boston, MA]

\vskip\baselineskip

% \vfill\break

\bf Some sound advice from Marcus Aurelius's \emph{τὰ πρός σεαυτόν}.

\rg
Ὅ τί ποτε τοῦτό εἰμι, σαρκία ἐστὶ καὶ πνευμάτιον καὶ τὸ ἡγεμονικόν. τῶν μὲν σαρκίων καταφρόνησον˙ λύθρος καὶ ὀστάρια καὶ κροκύφαντος, ἐκ νεύρων, φλεβίων, ἀρτηριῶν πλεγμάτιον. θέασαι δὲ καὶ τὸ πνεῦμα ὁποῖόν τί ἐστιν˙ ἄνεμος, οὐδὲ ἀεὶ τὸ αὐτό, ἀλλὰ πάσης ὥρας ἐξεμούμενον καὶ πάλιν ῥοφούμενον. τρίτον οὖν ἐστι τὸ ἡγεμονικόν. \emph{ἄφες τὰ βιβλία}˙ μηκέτι σπῶ˙ οὐ δέδοται. ἀλλ ὡς ἤδη ἀποθνῄσκων ὧδε ἐπινοήθητι˙ γέρων εἶ˙ μηκέτι τοῦτο ἐάσῃς δουλεῦσαι, μηκέτι καθ ὁρμὴν ἀκοινώνητον νευροσπαστηθῆναι, μηκέτι τὸ εἱμαρμένον ἢ παρὸν δυσχερᾶναι ἢ μέλλον ὑπιδέσθαι.

Τὰ τῶν θεῶν προνοίας μεστά. τὰ τῆς τύχης οὐκ ἄνευ φύσεως ἢ συγκλώσεως καὶ ἐπιπλοκῆς τῶν προνοίᾳ διοικουμένων. πάντα ἐκεῖθεν ῥεῖ˙ πρόσεστι δὲ τὸ ἀναγκαῖον καὶ τὸ τῷ ὅλῳ κόσμῳ συμφέρον, οǷ μέρος εἶ. παντὶ δὲ φύσεως μέρει ἀγαθόν, ὃ φέρει ἡ τοῦ ὅλου φύσις καὶ ὃ ἐκείνης ἐστὶ σωστικόν. σῴζουσι δὲ κόσμον, ὥσπερ αἱ τῶν στοιχείων, οὕτως καὶ αἱ τῶν συγκριμάτων μεταβολαί. \emph{ταῦτά σοι ἀρκείτω˙ ἀεὶ δόγματα ἔστω. τὴν δὲ τῶν βιβλίων δίψαν ῥῖψον, ἵνα μὴ γογγύζων ἀποθάνῃς, ἀλλὰ ἵλεως ἀληθῶς καὶ ἀπὸ καρδίας εὐχάριστος τοῖς θεοῖς}.

\vskip\baselineskip

\it In case that was all Greek to you, then rather read: 

\rm

Кио айн ми финфине естас, ми консистас де карно, ла спиро кай ла анимо. \emph{Форĵету виайн либройн}. Не есту дистрата - тио не естас пермесата; сед квазаŭ ви ям мортас, абомену ла карнон. Ĝи естас ненио кром санго, остетой кай рето де нервой, артериой кай вейной. Припенсу анкаŭ ла спирон. Киа материо естас ĝи? Аеро, киу не ĉиам естас ла сама, сед дум уну моменто естас вомата кай инспирата. До ла триан атентигу, ла анимон. Компрену, ке ви естас олдуло. Екде нун ви нек алласу, ке виа анимо фариĝас славон, нек алласу, ке ви каптиĝас де контраŭсоциай атакой, нек суферу де виа цирцумстанцаро нек тиму ла естонтецон.

Кио девенас де ла диой, пленас э провиденцо. Кио девенас де ŝанцо естас парто де натуро кай интертексатас кун тиой, киой девенас де провиденцо. Де провиденцо флуас ĉиой. Ĉиой екзистас про утило кай про тио, ке ĝи естас плей бона пор ла универсо, де киу ви естас парто. Кио естас бона пор ла тута универсо, кай кио субтенас ла натурон де ла универсо, анкаŭ естас бона пор парто де ла универсо. \emph{Есту контента кун тиуй максимой. Сед ĉесу соифи пор либрой пор ке ви алвену мортон не мурмуранте сед вере, транквиле кай данкеме ал ла диой ел виа анимо}.

\bye 

Kio ajn mi finfine estas, mi konsistas de karno, la spiro kaj la animo. \emph{Forĵetu viajn librojn}. Ne estu distrata - tio ne estas permesata; sed kvazaŭ vi jam mortas, abomenu la karnon. Ĝi estas nenio krom sango, ostetoj kaj reto de nervoj, arterioj kaj vejnoj. Pripensu ankaŭ la spiron. Kia materio estas ĝi? Aero, kiu ne ĉiam estas la sama, sed dum unu momento estas vomata kaj inspirata. Do la trian atentigu, la animon. Komprenu, ke vi estas oldulo. Ekde nun vi nek allasu, ke via animo fariĝas slavon, nek allasu, ke vi kaptiĝas de kontraŭsociaj atakoj, nek suferu de via circumstancaro nek timu la estontecon.

Kio devenas de la dioj, plenas je providenco. Kio devenas de ŝanco estas parto de naturo kaj interteksatas kun tioj, kioj devenas de providenco. De providenco fluas ĉioj. Ĉioj ekzistas pro utilo kaj pro tio, ke ĝi estas plej bona por la universo, de kiu vi estas parto. Kio estas bona por la tuta universo, kaj kio subtenas la naturon de la universo, ankaŭ estas bona por parto de la universo. \emph{Estu kontenta kun tiuj maksimoj. Sed ĉesu soifi por libroj por ke vi alvenu morton ne murmurante sed vere, trankvile kaj dankeme al la dioj el via animo}.

\bye